\section{Audio System}
\label{dmx_section}
Like every other I/O system in the engine, the audio is abstracted behind an interface. To fulfill \doom's core expectations, such a system has to implement at least twenty functions covering sound effects, music, and also the timer. Here are a few.\\
\par
 \begin{figure}[H]
\centering  
\begin{tabularx}{\textwidth}{ L{0.57}  L{1.43}}
  \toprule
  \textbf{Method} &  \textbf{Usage}\\

  \toprule 
  I\_StartupSound & Initialize audio system, detect audio hardware\\
  I\_SetChannels & Set number of channels and sample rate\\
  \toprule 
   
I\_RegisterSong & Upload a music lump and get back an ID.\\
I\_SetMusicVolume & Self explanatory.\\
I\_PlaySong & Play Song\\
I\_PauseSong & ...hm, Pause Song\\
I\_ResumeSong & Mysterious function with unknown effects.\\
I\_StopSong & Maybe this table was not a good idea after all.\\
I\_UnRegisterSong & Free music using ID obtained in \cw{I\_RegisterSong}.\\




  \toprule 
I\_GetSfxLumpNum & Upload audio sample from WAD and return an ID.\\
I\_SetSfxVolume & If you read this, you are a real human being and a real hero.\\
I\_StartSound & Start playing SFX sample.\\
I\_StopSound & Stop playing SFX sample and free it.\\
I\_SoundIsPlaying & Test if SFX is playing.\\
I\_UpdateSoundParams & Set pitch, left/right position and volume.\\

  \toprule 
  
I\_StartupTimer & On DOS, triggers audio system to hook into the Intel 8259 PIC. No-op on other platforms.\\
I\_ShutdownTimer & On DOS, remove the hook.  No effect on other platforms.\\

   \toprule
\end{tabularx}
\caption{\doom{}'s audio system interface}
\end{figure}



On NeXTSTEP this system was never a problem since only the timer function had to be implemented. On the PC side however, the game engine had to have sound effects and music. One difficulty was the fragmentation of the sound card market which had increased exponentially since Wolfenstein 3D. The previous title required tremendous effort just to support four types of sound card. Two years later there were more than fifteen available, all with different bugs, quirks and technologies.\\
\par
To make things worse, the departure of Jason Blochowiak had left id Software with both a shortage of expertise and enthusiasm toward audio. They solved the problem by throwing money at it and licensed the DMX library. For the price of its license, the library authored by Paul J. Radek offered an all-in-one audio solution. With support for all major sound cards, a convenient means to detect them, support for many sound and music formats, and an easy way to integrate with any game engine, DMX was a perfect fit which undoubtedly saved id several months of development.





The abstraction layer provided by DMX was a colossal task. The comment before the function in charge of detecting the hardware leaves no ambiguity as to how much of a burden it was to tackle this problem.\\
\par
\ccode{whypcssucks.c}\\
\par
With DMX backing the sound system, ten audio chipsets were supported.\\
\par
\ccode{cardenum_t.c}\\
\par
Running on an operating system supporting neither threads nor processes, there was seemingly no way to generate both the video and audio output simultaneously. Attentive readers will have noticed the hardware section mentioned two chipsets: the i8259 (Programmable Interrupt Controller: PIC) and i8254 (Programmable Interval Timer: PIT). Together they can be set up to interrupt the engine's execution and call into DMX's routines\footnote{The way the PIC and PIT interact is extensively detailed in "Game Engine Black Book: Wolfenstein 3D".}.\\
\par
Upon initialization, DMX installs itself as an interrupt handler to be called by the PIC and PIT at 140Hz. Upon awakening, DMX's interrupt handler takes care of feeding the audio device with music and sound effect data provided by the engine.\\
\par
Since it was tied to a timer, DMX was also in charge of the game engine's heartbeat via a variable called \cw{ticcount}. Everything in \doom{} uses that variable to pace itself.
\par


\scaleddrawing{0.95}{sound_manager_architecture}{DMX architecture}

With such an architecture there are two systems executing pseudo-concurrently but the audio has more constraints than the video. When an interrupt is triggered, DMX has only a few milliseconds to refill the sound card's buffers and go back to sleep. If DMX is too slow, it can delay video rendering or mask other interrupts.\\
\par
 This explains why allocation of audio assets gets special treatment in the zone allocator. The audio system has no time to recover from a memory miss (nor could it, since it has no access to the WAD or memory allocator). Audio data must be ready immediately whenever it is needed.

\subsection{Audio Data: Formats and Lumps}
The WAD archive contains hundreds of lumps of five types to feed to DMX. \cw{DP*} lumps are for PC speaker sound effects, \cw{DS*} lumps are for PCM sound effects, \cw{D\_*} lumps are for music tracks. There is also a \cw{GENMIDI} lump, and one \cw{DMXGUS} lump detailed later.\\
\par
PC speaker audio was meant for PCs without a sound card. The PC speaker was capable of square waves and meant to emit boot-up diagnostic noises. \doom{} found ways to make it generate something less irritating by changing the square wave frequency every 1/140th of a second\footnote{The PC Speaker is detailed in "Game Engine Black Book: Wolfenstein 3D".}. The data rate is one byte per 1/140th of a second, describing the square-wave frequency to set.\\
\par
The digitized audio samples used in \doom{} are 8-bit, 11025 Hz, mono PCM streams. PCM is a simple audio format that requires little explanation. Each byte in the stream is a moment in time that describes the amplitude of a waveform and indicates where the speaker cone should be positioned. The sound card simply translates the byte into a voltage to the magnet driving the cone position, which happens at a rate of 11025 times per second.\\
\par
Music data is stored in \cw{MUS}, a format similar to standard MIDI but slightly more compact. The format allows eight channels for instruments and a ninth for the drums. The MUS format describes a series of precisely timed events for each channel which specify which instrument should play a note and at what pitch. Note that MUS only describes what to do and when, not how to do it. Each channel refers to the note of an instrument; the instrument itself is described elsewhere in a data structure known as the "instrument bank".\\
\par
The instrument bank is stored in the \cw{GENMIDI} lump. It is aimed at SoundBlaster-compatible sound cards based on the OPL2 chip which synthesize music using Frequency Modulation. The lump describes how to play the note of each instrument in the MIDI instrument set. There are 175 entries in \cw{GENMIDI}, one for each of the 128 standard General MIDI instruments and 47 percussion effects. Each entry describes how to set up a channel in order to emulate an instrument. A channel is made of two cells, where one cell is used as carrier and the other as modulator. For each cell, attack, decay, sustain, release, harmonic type, and waveforms can be selected by the musician. Instrument banks were 90s musicians' "secret sauce" and \doom's General MIDI emulated FM patch set were known for being particularly good\footnote{Source: "The dark and forgotten art of GENMIDI" in Freedoom's documentation.}.\\
\par
The \cw{DMXGUS} lump played the same role as \cw{GENMIDI} but for the Gravis Ultrasound card. It enabled the GUS to play MUS music notes, not with an FM synthesizer but with the PCM samples provided with GUS drivers. It is a simple lump mapping MIDI instruments to GUS instruments with special rules for RAM allocation depending on the amount of RAM installed on the board (256KiB, 512KiB, 768KiB, and 1024KiB).\pagebreak



Dealing with so much complexity proved difficult for DMX. Some of the craziness Paul Radek had to deal with surfaced in one of his post on \cw{usenet}\footnote{Forum thread: "Gravis Ultrasound - Hardware Mixing Game List".}.\\





\fq{All of the sound effects are now mixed in software, rather than on the GUS hardware. Why, you ask? Because of several reasons. First, is that the GF1 chip has a minimal ramp time that is much to long for very sharp effects. Second, because loading of the MUSIC patches uses all of the GUS memory, I had to DMA all eight sound effects to the card when played. This intern exposed a bug in the GF1 chip that Gravis did not find until my code started to beat on it. The bug would cause the bus to freeze and any program with it. The workaround is to keep DMA activities to a minimum by mixing in software and transferring only 1 channel to the GUS. But since the GF1 can't support auto-initialize DMA, and because the only way to play interleaved data on the card is to set two voices pointing into a single patch and setting the frequency so the every other sample is skipped, you don't get the benefit of sample smoothing from the GF1.\\
\par
Sorry, but that's the way it has to be :(\\
}
{Paul Radek, Digital Expressions, Inc.}\\
\par

The library evolved during the development of \doom. Sometimes API changes introduced bugs. Gravis UltraSound support was broken with v1.666. Support for the Audio Spectrum was also broken so users of the card had to fall back on (poor) SoundBlaster emulation\footnote{Source: John Romero post on \cw{alt.games.doom}.}.\\
\par
The problem was partly due to poor API practices on DMX's side and partly because of hasty adjustments on id Software's part. Most issues were ultimately fixed except for one major bug which made it to gamers. The engine was originally supposed to emit sounds at random pitches to avoid monotony. To this effect, DMX function \cw{SFX\_PlayPatch} was used.\\
\par
\ccode{dmx_before.c}





In early versions it worked as intended but then the DMX API was modified in an incompatible way. It may not be immediately apparent but the parameters to \cw{SFX\_PlayPatch} are swapped.\\
\par
\ccode{dmx_after.c}\\
\par
 Invocation sites were never adjusted and the game shipped without the random pitch feature, instead randomly balancing sound between the left and right channel.\\
\par
\ccode{I_StartSound.c}\\
\par
In retrospect John Carmack regretted using DMX because it led to issues when open sourcing the game engine (it is unknown if Paul Radek was unwilling to open source DMX or if id Software was unwilling to negotiate with him).\\
\par
\fq{
Our biggest mistake during DOOM development was the contracting of an  
outside party to do dos sound drivers.  Because we had this black box  
functionality coming, I didn't simulate it under NS.  BAAAAAD mistake.   
All future work will be entirely developed under NS, with only DMA  
buffer flipping being the hardware layer.  We will probably also run  
midi under NS for music (which will be dynamically tuned to the game  
situation in Quake).}{John Carmack}\\
\par
One can still find archived Usenet posts from \cw{alt.games.doom} where John Romero's comments hint that the relationship between Radek and id Software was suffering as the game neared its release date\footnote{Source: John Romero post on \cw{alt.games.doom}.}. Since the tone was less than cordial, it is left as an exercise to the reader if they want to dig these posts out. I do not recommend it.
% \par
% \fq{Hey, everybody!  Thanks a lot for downloading DOOM II -- that's really "cool".   
% We love it when we send evaluation copies to magazine editors and some piece of  
% shit uploads the thing.  The DOOM II out there is the master copy we sent to GT  
% Interactive, our distributor.  If your GUS sound doesn't work, it's probably  
% because your IRQ is > 7.  Our sound code dork broke part of the sound support  
% while he was "expanding" it, so IRQs > 7 don't work, Pro Audio Spectrum owners  
% now have to run their SB-emulation driver (no more native support), etc.  The  
% sound guy is a real shithead.  All these fuckups are in v1.666 as well --  
% sorry.  We have NO time to wait and hope that sound-dork fixes these problems,  
% as he's demonstrated in the past year and a half that he's incompetent.}{John Romero (alt.games.doom)}



\section{Sound Propagation}
How enemies react to sounds makes a tremendous impact on how smart the A.I. is perceived as being. id Software made sure sound propagation was realistic. When a shot is fired in a sector, a flood fill algorithm propagates the noise.\\
\par
\ccode{P_RecursiveSound.c}\\

\begin{wrapfigure}[10]{r}{0.48\textwidth}
\centering
\includegraphics[width=.48\textwidth]{drawings/sound_blockers.pdf}
\end{wrapfigure}
  The sector/portal format is leveraged, starting from the player's sector and flood-filling into adjacent sectors via portals (two-sided lines). Sound is stopped when either a door is encountered or when two "sound blocker" lines have been crossed. Level E3M9 features a section with 38 monsters where the A.I. cost is lowered by three (red) blocker lines, ensuring some monsters remain dormant.\\
\par
Notice how function \cw{P\_RecursiveSound} mentions "waking up all monsters" but never iterates over the list of things in the level. This is a speed-up trick aimed at avoiding an expensive search to find all monsters to be awakened in each sector. Monsters always look up for the current sector's \cw{soundtarget} to pick a target. By simply assigning a value to \cw{sec->soundtarget}, all monsters in \cw{sec} automatically acquire the same target.\\
\par
\rawdrawing{E1M1_audio_sections}
\par
\subsubsection{Ambushing}
Map designers wanted to have sneaky monsters who could hide and wait to jump out at players. This is achieved via the MF\_AMBUSH flag assigned to monsters. It doesn't make monsters deaf, rather it makes them not seek the player until they make visual contact.


\subsubsection{Super Ambushing}
Sound propagation was used inventively in level E1M9 for its super ambush.  The designers wanted the player to be swarmed with monsters teleporting, seemingly from a hellish dimension, as soon as the player walked across the center of a pentagram.\\
\par
 Without a scripting language available that was next to impossible to implement. As we'll see in the A.I. section, monsters are state machines only able to do three things: stay dormant, pursue a target, and when they bump into walls and things, change direction.\\
\par
\begin{wrapfigure}[24]{r}{0.52\textwidth}
\centering
\includegraphics[width=.52\textwidth]{drawings/E1M9_sides.pdf}
\end{wrapfigure}

\par

To achieve the effect, they created an inaccessible "monster pool" room next to where they wanted the super ambush and filled it with monsters (red circles in the diagram). In the south-east corner of the hidden room, they placed a teleporter, protected by walls.\\
\par
Then they created a very tiny pipe to allow sound to flow between rooms, so that monsters from the "pool" room would wake up and try to reach the player. Without another way to get there (the pipe being at ceiling level and the monsters being too big to fit through it), monsters will wander in circles with a tendency to move towards the player's location (south-east).\\
\par
The last piece of the super ambush trap was to have four tripwires around the center of the "pentagram" which lowered the walls around the teleporter. A tempting bait was placed on it (health and ammunition) to make sure the player would go there.\\
\par

As soon as the player crosses the trigger in the "star" room, the walls around the teleporter lower, monsters move towards it and swarm the player. \\
\par
\trivia{Comments in the code suggest that monsters screaming awaken other monsters but this is not the case. It is unknown why this feature was cut. Maybe it was too buggy or too expensive.}







John Romero himself described how they called these sound conduits, "pipes".\\
\par
\fq{We used sound zones in Wolfenstein 3D as another way to alert enemies to your presence. In DOOM, we did the same thing but used sectors as the conduits of audio travel. This was a really important part of making the game scary, as sound could leak all over the place and alert demons. You might see lots of little sector pipes that connect sectors together just to alert monsters-sectors that you'd never see because we put them way up high in the corner of a room. So, we paid a lot of attention to the sound flooding.}{John Romero, p29 in Scarydarkfast}\\
\par
The audio pipe for the E1M9 super ambush is so well-hidden in the ceiling that it is very hard to notice with the vanilla engine. The room is dark and in the distance, the tiny black rectangle blends in the obscurity. Opening the map with an editor such as SLADE that allows looking up and down reveals the mechanism.\\
\par
\cfullimage{sound_conduit.png}{}
\par
In figure \ref{sound_conduit.png} the audio conduit is visible in the upper right corner.
% In figure \ref{sound_conduit.png}, the audio conduit to awaken the monsters and make them go to the teleporter is visible in the upper right corner (screenshot done with the amazing Slade3 Node Editor).
