\begin{wrapfigure}[10]{r}{0.30\textwidth}{
\centering \scaledimage{0.30}{saturn_logo.png}}
\end{wrapfigure}
Development of the Sega Saturn started in June 1992\footnote{Source: "Console Wars: Sega, Nintendo, and the Battle That Defined a Generation".} as a replacement for the insanely popular yet aging Genesis. At this point in time, the Genesis had sold more than 30 million units and had a "cool" image among the 15-25 range - an image built with many good games and massive TV advertising campaigns. For Sega, It was a colossal yet mandatory undertaking to at least match its predecessor.\\
\label{saturn_port}
\par
After two years of hard work, Sega demonstrated a prototype Saturn during the Tokyo Toy Show in June 1994. Unknown to them, it would cause even more damage than the 32X, ruin Sega International's image, and sell poorly.\\
\par
During its development, Sega worked in partnership with Hitachi to develop a new CPU tailored to its needs. The joint venture resulted in the "SuperH RISC Engine" (a.k.a SH-2) at the end of 1993 which Sega used in dual configuration as foundation for the Saturn.\\
\par
 On the graphics side, one video display processor (VDP) was to do most of the job. However reports of the PlayStation's capabilities prompted Sega to add a second VDP to improve the system's 2D performance and texture-mapping.\\
\par
\fullimage{consoles/Saturn.png}\\

\vspace{-10pt}
Sega managed to release its console before the dreaded PSX and sales in Japan were initially promising with games such as Daytona USA and especially Virtua Fighter being well received. The initial success had a lot to do with Virtua Fighter which was by far the most popular arcade game in Japan at the time\footnote{Source: "Virtua Fighter Mania". GamePro. No. 89. February 1996.  p28.}.\\
\par
Beating Sony came at a great price and the result seemed rushed. During E3 1995 in Los Angeles, Sega CEO Tom Kalinske surprise-announced that the Saturn would be available the very same day. Even their supplier did not know about this and the console was out of stock rapidly. Another consequence of the rush was that only six games were available upon launch. Panzer Dragoon, which could have made for a perfect flagship title, missed its deadline\footnote{Source: "The Making Of... Panzer Dragoon Saga", \cw{nowgamer.com}.}.\\
\par The machine was also difficult to program. 3D was achieved in a similar way to the 3DO. Programmers had to deal with 2D quads which could be distorted in screen space to poorly fake perspective. It was not a matter of not trying hard, the hardware was complex.\\
\par
\fq{Currently we only use the Master SH2, the slave SH2 will be used when we get around to figuring out how.}{Mick West 1995 Saturn development journal}\\
\par
Worldwide, the platform had a lukewarm reception. And then the beast was unleashed.\\
\par
 Two weeks after release, the PSX came out with Ridge Racer and took the world by storm\footnote{The PlayStation outsold the Saturn by a factor of three.}. Not only was the Saturn more expensive (\$399) than the PSX (\$299), games such as Daytona USA which used to look good now had blatant issues when put side-by-side with Ridge Racer. The lower framerate, polygon pop-up and letter-boxed presentation begged for mercy. To add more pressure, in June 1996 the market welcomed another competitor with the Nintendo 64.\\
\par
% Ironically Sega's next console, the DreamCast addressed all issues.  It was easy to program, ran on Windows and was very well liked by developers.
As a symbol of the failure, Sonic X-treme, which would have been the first fully 3D Sonic the Hedgehog game and an original Sonic game for the Saturn was canceled after three years in an apocalyptic development which almost took the life of its producer Mike Wallis.\\
\par
In 1998 Sega released the Dreamcast. Well-received by developers (except for Electronic Arts who snubbed it completely), the system suffered from an early copy-protection hack and competition from the Xbox and the PS2. In late 2001, on the verge of bankruptcy, Sega withdrew from the hardware business to focus on its "Virtua" line of products.


\subsection{Programming the Saturn}
Programming the Saturn was a difficult task. The programmer's manual is broken down into eight voluminous manuals requiring repeat reading sessions to build a mental image of the flow of data. The diagram in figure \ref{arch_saturn} gives some idea of the daunting effort required to coordinate eight chips.\\
\par
Main programming was done via the two SH-2 processors connected to 2.0MiB of shared RAM. One SH-2 was deemed the master and the other the slave. In the common configuration, the slave was intended to be used as a helper for parallelizable tasks. Communication between chips (to indicate what to execute) had to be done via a tedious interrupt system. To deal with static and global variables the programmer had to deal with mutexes and semaphores which were uncommon concepts for game console programmers. Because they were on the same bus, one had to wait for the other if both needed to access either to RAM or a peripheral on the system. Attempts to minimize the issue were made via a shared unified 4KiB cache that had little impact on the bottleneck.\\
\par
Audio was done via the SCSP (Saturn Custom Sound Processor) that piloted a sound processor (a Motorola 68000). The SCSP was to be configured to perform sound mixing in the dedicated 512 KiB of RAM, which was then picked up by the Sound Processor. The combination of these made it a powerful system able to synthesize instruments, play PCM sound and perform 3D effects/distortion. The chip also polled control inputs from the player and stored them in internal registers to be polled by the SH-2s.\\
\par
Graphics programming was done via two chips called VDP1 and VDP2. The VDP1 was a hardware-accelerated quad renderer. It had the particularity to use forward texture-mapping which is very efficient when rendering sprites (like in 2D games) but not so much when magnifying or minifying textures (like in 3D games). Rendering was done targeting layers which once ready were picked up by the the VDP2, composited according to their priority and transparency settings, and sends to the TV. Note that the two chips work in parallel. While the VDP1 works on the next frame, the VDP2 finishes the previous one.\\

\par
Access to the CD-ROM was done via a driver piloting the SH-1 processor. The double-speed unit could read at 150 KiB/s but average access time was 300 ms. To compensate, the SH-1 stored data to a 512KiB buffer. Based on the abysmal access time, programmers were instructed to request data well in advance.\\ 
\par
To control all these components and transfer data between systems, a seventh chip called the SCU (System Control Unit) acted as DMA controller, DSP and bus controller. The DSP was able to perform matrix transformations and write the result directly in the VDP RAM.\\
\par


\trivia{Reading the programmer manual in detail reveals that each component can somehow interact with each others' RAM. This made debugging very difficult.}
\pagebreak

\scaleddrawing{1}{arch_saturn}{Sega manual: "Introduction to Saturn Game Development", April '94 }




\cfullimage{Sega-Saturn-Motherboard.png}{Sega Saturn motherboard}
\par
Opening a Sega Saturn and taking a look at the motherboard reveals close to twenty chips.\\
\par
\circled{1} 32-bit 28.6 MHz SH-2, 
\circled{2} 32-bit 28.6 MHz SH-2, 
\circled{3} VDP2, 
\circled{4} The YMF292, aka SCSP (Saturn Custom Sound Processor), 
\circled{5} SCU DSP Math coprocessor @ 14.31818 MHz, 
\circled{6} BIOS, 
\circled{7} SMPC (System Management \& Peripheral Control), 
\circled{8} Motorola 68CE00, 
\circled{9} 32 KiB Battery-backed SRAM, 
\circled{A} 4 MiB RAM (2MiB RAM + 1.5MiB VRAM + 540KiB Audio RAM), 
\circled{B} VDP1, 
\circled{C} Hitachi CD-ROM I/O data controller, 
\circled{D} 32-bit 20 MHz SH1  microcontroller with 64k internal ROM, 
\circled{E} Two controllers connectors, 
\circled{F} A/V OUT socket,  
\circled{G} Sega Communication socket,  
\circled{H} Cart slot (RAM extender requested for "X-Men vs Street Fighter"), 
\circled{I} CD-ROM connector.






\rawdrawing{saturn_motherboard}
Despite its issues and ill-timed release, it is a bitter feeling to see what happened to the Saturn and seeing it considered a failure. Over its four years of life, the platform managed to host amazing technical and entertaining games such as Radiant Silvergun, Grandia, Sega Rally Championship, Virtua Fighter 2, Panzer Dragoon Saga, Guardian Heroes, NiGHTS into Dreams, Panzer Dragoon II Zwei and Virtua Cop. Unfortunately, \doom{} would not be part of the previous list.\\
\par
\fq{After years of waiting, Doom finally arrives on Saturn. Unfortunately it is a breath-takingly bad conversion of a classic game.}{Sega Saturn Magazine \#16, February 1997}




























\subsection{\doom{} on Saturn}
% The port to the Saturn was done by Rage Software on a very tight schedule. Like most ports, it is based on the Jaguar assets. The console had the same technical shortcomings when it comes to texture sampling, as Jim Bagley remembers\footnote{RetroGamer \#134.}:\\

The port to the Saturn was done by Rage Software on a very tight schedule. The graphic part of the engine was done via the VDP1 writing quads into three separate layers (a floors, ceiling, and walls layer, a things layer, and a status bar layer) which were combined by the VDP2 and sent to the TV. The resulting framerate was outstanding compared to other ports but the lack of perspective correct texturing ended up disturbing Jim Bagley's plans\footnote{RetroGamer \#134.}:\\
\vspace{10pt}
\par
\fq{When I started the project, I had to do a demo for id Software to approve. I started by extracting all the levels and audio and textures from the WAD files and made my own Saturn version of this, then got an early version of the renderer working using the 3D hardware. This got sent off and a couple days later I got a call from John Carmack, who stipulated that under no circumstances could I use the 3D hardware to draw the screen. I had to use the processor like the PC. Thankfully I enjoy challenges, so it turned out to be a really enjoyable project, using both SH2s to render the display like the PC did it, using the 68000 to orchestrate them both.\\
\par
However, it kneecapped the game and the speed-framerate suffered greatly.}{Jim Bagley for RetroGamer \#134}\\
\par
Years later, by 2014, Carmack had reconsidered.\\
\par
\fq{I hated affine texture swim and integral quad verts, but in hindsight, I probably should have let experiment.}{John Carmack}\\
\par
In the end, the VDP1 hardware-accelerated 60 FPS-capable engine was tossed.\\
\par
 Due to time constraints, Jim did not have the time to change the renderer to work with pixel-wide triangles like the PlayStation. Upon shipping, the game managed a framerate that could reach 20 FPS but most of the time this dropped to the single digits at a full screen resolution of 281x235. To compensate for the low framerate, Jim Bagley made the decision to slow down all movements, a move that enraged the playing community.\\
\par
\trivia{Another effect of the rushed schedule was a bug with the audio system that made all sound effects panned left. Players had to play in mono to hear from both speakers.\footnote{Digital Foundry: "Every Console Port Tested and Analysed!".}}\\
\par






\fullimage{doom_saturn1.png}\\
\par
In the screenshot above, notice how E1M1 is the same as other console versions (all based on the initial work for the Jaguar). The status bar however welcomed a makeover.\\
\par
What "knee-capped" the project was the walls, ceilings and floors rendition (accounting for most of the computing cost) which ended up being software-rendered via the SH-2s while the status bar and the things (such as monsters and walls featuring transparent parts) were hardware-accelerated via the VDP1\footnote{Digital Foundry: "Every Console Port Tested and Analysed!".}. When all three layers were ready, the VDP2 composited the three layers toward the TV while the VDP1 started to render the next frame.\\
\par 
Translucency was done in a peculiar way for which the details constitute a testament to the complexity of the machine. Both the VDP1 and the VDP2 were capable of "half-transparency", a term referring to equally blending source and target. However the VDP1 only supported transparency in 15-bit color while the VDP2 only supported transparency in indexed mode. As a result, you could either have sprites be transparent with regards to each other or transparent with regards to the VDP2 background layers, but not both.\\
\par
This limitation was a big problem to render "Spectre" enemies. If the VDP1 marked the Spectre pixels "half-transparent", they would properly render over the background layer. However they would also have "swallowed" any other sprites possibly standing behind them, generating incorrect scenes\footnote{Source: "The Sega Saturn and Transparency" by Matt Greer.}.\\
\par
Sega designers were well-aware of the limitation of the VDPs. To palliate the problem they introduced the concept of "mesh" sprites which were rendered opaque by the VDP1 but only every other pixels.\\
\par
\cfullimage{doom_saturn4.png}{A Spectre enemy rendered as a "mesh".}
\par
The pixel-perfect screenshot, especially the zoomed-in version next page, may look crude at first sight. However, you have to keep in mind the composite interleaved display system which ended up blending everything together. Even though the visual result was remarkably convincing, this magic-trick did not survive the "HD" pixel-perfect era.\\
\par
\cfullimage{doom_saturn41.png}{Same scene, zoomed-in to show skipped pixels in the Spectre.}
